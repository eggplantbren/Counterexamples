\documentclass[a4paper, 12pt]{article}

\usepackage{graphicx}

\title{P-values behaving badly}
\author{Brendon J. Brewer}

\begin{document}

\maketitle

Criticizing P-values is a favorite pastime among Bayesians. No
``counterexamples'' repository would be complete without it.

\section{Voilating {\em more}}
I learned this example from Alex Etz ({\tt http://alexanderetz.com/}).
Imagine a drug company has an old drug which is known to cure
75\% of patients of a disease. They are developing a new drug
which they hope will be even more effective.
The effectiveness of the old drug is $\theta_{\rm old} = 0.75$ and the
effectiveness of the new drug is $\theta_{\rm new}$, which is unknown.
Consider three ``coarse-grained''
hypotheses about the value of $\theta_{\rm new}$:

\begin{eqnarray}
H_<: \theta_{\rm new} < 0.75\\
H_0: \theta_{\rm new} = 0.75\\
H_>: \theta_{\rm new} > 0.75
\end{eqnarray}

To measure $\theta_{\rm new}$, the company tests the drug on $N=50$ patients
and counts the number, $x$, who recover. The probability distribution for
$x$ given $\theta_{\rm new}$ (and $N$) is Binomial(50, $\theta_{\rm new}$):

\begin{eqnarray}
p(x | \theta_{\rm new}) &=&
\left(\begin{array}{cc}N \\ x\end{array}\right)
\theta_{\rm new}^x \left(1 - \theta_{\rm new}\right)^{N - x}.
\end{eqnarray}


\begin{figure}[ht!]
\centering
\includegraphics[scale=0.5]{drug_example.pdf}
\caption{\label{fig:drug_example}}
\end{figure}

\end{document}

