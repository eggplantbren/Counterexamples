\documentclass[a4paper, 12pt]{article}

\usepackage{graphicx}

\title{P-values behaving badly}
\author{Brendon J. Brewer}

\begin{document}

\maketitle

Criticising P-values is a favorite pastime among Bayesians. No
``counterexamples'' repository would be complete without it.

\section{{\em Or} implies {\em more}}
I learned this example from Alex Etz ({\tt http://alexanderetz.com/}).
Consider two propositions, $A$ and $B$, which you are trying to reason about.
If you have some judgment about how plausible $A$ is, what would you think
about the proposition $A \vee B$ (read $A$ {\bf or} $B$), which would be true
if either one of $A$ or $B$ is true (or both)? 



In probability theory (i.e. Bayesian statistics), the sum rule relates the
probability of a join (logical {\bf or}) like so:
\begin{eqnarray}
P(A \vee B | I) &=& P(A | I) + P(B | I) - P(A, B | I)\\
                &=& P(A | I) + P(B | I) - P(A | I)P(B | A, I)\\
                &=& P(A | I)\left[1 - P(B | A,I)\right] + P(B | I)\\
                &\geq& P(B | I). 
\end{eqnarray}
i.e. {\em or} implies {\em more}.


Imagine a drug company has an old drug which is known to cure
75\% of patients of a disease. They are developing a new drug
which they hope will be even more effective.
The effectiveness of the old drug is $\theta_{\rm old} = 0.75$ and the
effectiveness of the new drug is $\theta_{\rm new}$, which is unknown.
Consider three ``coarse-grained''
hypotheses about the value of $\theta_{\rm new}$. The new drug's
effectiveness could be worse than, equal to, or better than that of the old
drug:

\begin{equation}
\begin{array}{lcl}
H_{\rm worse}:  &  &\theta_{\rm new} < 0.75\\
H_0:            &  &\theta_{\rm new} = 0.75\\
H_{\rm better}: &  &\theta_{\rm new} > 0.75
\end{array}
\end{equation}

The middle proposition, $H_0$,
is the ``null hypothesis'' that the new drug and the
old drug are identical in terms of effectiveness. Classical statistical tests
are used to quantify the strength of the evidence against $H_0$. In reality,
it's best to consider the full set of hypotheses about what the value of
$\theta_{\rm new}$ might be, rather than boiling it down to these three
scenarios. This is related to the idea
that ``statistical significance'' and ``practical significance'' are not the
same. For example, $\theta_{\rm new}$ could be 0.750002, which is
technically ``better'' than the old drug, but not by an amount that's likely
to matter.

To measure $\theta_{\rm new}$, the company tests the drug on $N=50$ patients
and counts the number, $x$, who recover. The probability distribution for
$x$ given $\theta_{\rm new}$ (and $N$) is Binomial(50, $\theta_{\rm new}$):

\begin{eqnarray}
p(x | \theta_{\rm new}) &=&
\left(\begin{array}{cc}N \\ x\end{array}\right)
\theta_{\rm new}^x \left(1 - \theta_{\rm new}\right)^{N - x}.
\end{eqnarray}

\begin{figure}[ht!]
\centering
\includegraphics[scale=0.5]{drug_example.pdf}
\caption{\label{fig:drug_example}}
\end{figure}




\end{document}

